\documentclass[12pt]{article}

\usepackage[utf8]{inputenc}
\usepackage[T2A]{fontenc}
\usepackage[english, russian]{babel}
\usepackage{musixtex}

\usepackage[a4paper,margin=2cm]{geometry}

\begin{document}
	
	\centering
	\large{Лабораторная работа №6, дополнительное задание 2}
	
	Елисеев Константин Иванович, P3108
	\vspace{0.5cm}
	
	Номер варианта $= 1 + ((7 * 10) \bmod 27) = 17$
	
	Гимн Австралии
	
	\noindent\rule{\textwidth}{1pt}
	\vspace{0.5cm}
	
	\begin{music}
		\parindent10mm
		\nobarnumbers % убрать нумерацию окончания тактов
		
		%\setstaffs1{2} % установить набор ключей
		\generalmeter{\meterfrac44} % задать величину такта
		%\setclef1{\bass} % установить басовый ключ
		\generalsignature{3} % кол-во диезей
		
		\startextract
			\Notes\qu{e}\en
			\bar{|}
			\Notes\qu{hece}\en
			\bar{|}
			\NOtesp\qup{h} \en
			\NOtes\zcu{h} \en
			\NOtesp\qu{h} \en
			\NOtesp\zql{j} \en
			\bar{|}
			\NOtesp\zhl{i} \en
			\Notes\qp \en % Четвертная тактовая пауза
			\Notes\qu{e}\en
			\bar{|}
			\Notes\qu{h}\en
			\Notes\qu{e}\en
			\Notes\qu{c}\en
			\Notes\qu{a}\en
			\bar
		\zendextract
			
		\startextract
			\elemskip 0.5\elemskip
			\NOtesp\qup{e} \en\NOtes\zcu{e} \en
			\NOtesp\qu{e} \en
			\NOtesp\zql{j} \en
			\bar{|}
			\NOtesp\zql{i} \en
			\NOtesp\qu{h} \en
			\NOtesp\qu{g} \en
			\NOtesp\qu{f} \en
			\bar{|}
			\Notes\zhu{e} \en
			\Notes\qp \en
			\NOtesp\qu{e} \en
			\bar{|}
			\NOtesp\qup{f} \en
			\NOtes\zcu{g} \en
			
		\zendextract
	\end{music}
	
\end{document}